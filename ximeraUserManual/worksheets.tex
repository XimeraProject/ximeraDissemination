\documentclass{ximera}

\title{Worksheets}

\author{Bart Snapp}

\begin{document}
\begin{abstract}
    We'll call a single document with several questions a worksheet.
\end{abstract}
\maketitle

A worksheet is a piece of paper with questions on it.
This section explores the different types of question types Ximera supports. 


\section{The \texttt{answer} command}
 
The basic way of including a answerable item in Ximera is to use the
\verb|\answer| command. The \verb|\answer| \textbf{must} be inside of
an environment.
 
\begin{verbatim}
\begin{question}
$3\times 2 = \answer{6}$
\end{question}
\end{verbatim}
 
Will produce the question:
 
\begin{question}
  $3\times 2 = \answer{6}$
\end{question}
 
In addition to numerical answers, we also support elementary functions:
 
\begin{verbatim}
    \begin{question}
         $ \frac{\partial}{\partial x} x^2\sin(y) =  \answer{2x\sin(y)}$
    \end{question}
\end{verbatim}
 
Produces:
 
\begin{question}
  $\frac{\partial}{\partial x} x^2\sin(y) = \answer{2x\sin(y)}$
\end{question}
 
\begin{remark}
Under the hood, Ximera is parsing the user input, producing a
function, and checking the user input function against ``answer'' at
$100$ different complex numbers, and seeing if the results are
``reasonably'' close to each other.  We compare the complex extensions
of these functions to circumvent domain issues.
\end{remark}
 
 
While \textbf{any} environment can contain the command \verb|\answer|,
there are four special environments: \verb|question|, \verb|exercise|,
\verb|problem|, \verb|exploration|. Each of these environments is the
same, except for the name. These environment interact with the
optional arguments in the documentclass in useful ways. We'll discuss
this later.
 
 
 
 
 
\section{Choice Answer Type}
 
\subsection{Multiple Choice}
 
\begin{verbatim}
\begin{question}
    Which of the following functions has a graph which is a parabola?
    \begin{multipleChoice}
        \choice[correct]{$y=x^2+3x-3$}
        \choice{$y = \frac{1}{x+2}$}
        \choice{$y=3x+1$}
    \end{multipleChoice}
\end{question}
\end{verbatim}
 
Produces:
 
\begin{question}
  Which of the following functions has a graph which is a parabola?
  \begin{multipleChoice}
    \choice[correct]{$y=x^2+3x-3$}
    \choice{$y = \frac{1}{x+2}$}
    \choice{$y=3x+1$}
  \end{multipleChoice}
\end{question}
 
\begin{remark}
  Multiple choice answers appear in the order you type them.
\end{remark}
 
 
\subsection{Select All}
 
\begin{verbatim}
\begin{question}
  Which of the following numbers are even?
  \begin{selectAll}
    \choice[correct]{$2$}
    \choice{$1$}
    \choice[correct]{$-4$}
    \choice[correct]{$0$}
  \end{selectAll}
\end{question}
\end{verbatim}
 
Produces:
 
\begin{question}
  Which of the following numbers are even?
  \begin{selectAll}
    \choice[correct]{$2$}
    \choice{$1$}
    \choice[correct]{$-4$}
    \choice[correct]{$0$}
  \end{selectAll}
\end{question}
 
\begin{remark}
  Select All answers appear in the order you type them.
\end{remark}
 
 
%% \section{Matrix Answer Type}
 
%% We also support matrices of expressions.
 
%% \begin{verbatim}
%% \begin{question}
%% Enter the matrix  \(\begin{bmatrix} x & y \\ xy & z+1 \end{bmatrix}\)
%%     \begin{matrixAnswer}
%%      correctMatrix = [['x','y'],['xy','z+1']]
%%     \end{matrixAnswer}
%% \end{question}
%% \end{verbatim}
 
%% \begin{question}
%%   Enter the matrix  \(\begin{bmatrix} x & y \\ xy & z+1 \end{bmatrix}\)
%%   \begin{matrixAnswer}
%%     correctMatrix = [['x','y'],['xy','z+1']]
%%   \end{matrixAnswer}
%% \end{question}
 
%% \begin{remark}
%%   The plus and minus buttons add and subtract columns or rows. 
%% \end{remark}
 
\section{free-response}
 
The free response environment gives students access to a \LaTeX\ editor.
 
\begin{verbatim}
\begin{question}
    Question goes here!
    \begin{freeResponse}
    This is the model solution %You don't actually need anything in between the begin and end line.
    \end{freeResponse}
\end{question}
\end{verbatim}
 
\begin{question}
    Question goes here!
    \begin{freeResponse}
    This is the model solution %You don't actually need anything in between the begin and end line.
    \end{freeResponse}
\end{question}
 
\begin{remark}
Clicking on \verb!View model solution! shows the user
whatever you typed in the  \verb!freeResponse! environment.
\end{remark}
 
\end{document}