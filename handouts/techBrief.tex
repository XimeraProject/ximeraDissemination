\documentclass{article}
\usepackage{tikz}
\usetikzlibrary{shapes}
\usepackage{standalone}
\usepackage[cm]{fullpage}
\usepackage{stix2}
\usepackage{gillius}
\tikzset{>=stealth}

\usepackage{multicol}

\usepackage[letterspace=-80]{microtype}
\begin{document}
\pagenumbering{gobble}
\sffamily\flushleft
\hspace{-.3cm}\begin{tikzpicture}%
    \node at (0,0) {\tikzset{draw/.append style={black}}
    \resizebox{!}{.05\textheight}{\input{../wormLogo/tikzLogo.tex}}};
    \node at (6.4,0) {\resizebox{!}{.05\textheight}{\lsstyle Tech\! Brief}};
\end{tikzpicture}\\
Ximera: Interactive Mathematics Education  \\[-.1cm] Resources for All\hfill \textsl{MathFest} 2024: August 7--9\\[.1cm]
\rule{\textwidth}{.1cm}\\[.1cm]\rmfamily
Email:~{\tt ximera@math.osu.edu} \hfill Website:~{\tt
https://github.com/ximeraProject/}\\[1cm]
\begin{multicols}{2}
    Ximera, pronounced ``chimera,'' is an
    open-source platform that provides tools for authoring and publishing
    (PDF and Online), open-source, interactive educational content, such
    as textbooks, assessments, and online courses.
    The Ximera Project is
    funded 2024-2026 with (no other external funding) by a U.S.\ Department of
    Education \$2,125,000
    Open Textbooks Pilot
    Program grant. In the
    past, the Ximera Project has also recieved support from NSF Grant
    DUE-1245433, the Shuttleworth Foundation, the Ohio State University
    Department of Mathematics, and the Affordable Learning Exchange at
    OSU.\\[.5cm]
    {\sffamily\bfseries Authors}\\
    With Ximera, authors use LaTeX to create their content. With this
    single source code, we generate different types of output: A PDF worksheet, an
    online interactive activity, and/or a PDF solution manual.
    % \begin{center}
    %     \includegraphics[width=.4\textwidth]{SimultaneousOutput.jpg}
    % \end{center}
    \begin{center}
        \begin{tikzpicture}
            \node at (-1.8,.2) {\resizebox{.65cm}{!}{\input{document.tex}}};
            \node at (-1.8,.2) {\small PDF};
            \node at (-2,0) {\resizebox{.65cm}{!}{\input{document.tex}}};
            \node at (-2,0) {\small PDF};
            \node at (-2.2,-.2) {\resizebox{.65cm}{!}{\input{document.tex}}};
            \node at (-2.2,-.2) {\small PDF};
            \draw[->] (-.6,0) -- (-1.4,0);
            \draw[->] (-.6,.2) -- (-1.4,.2);
            \draw[->] (-.6,-.2) -- (-1.4,-.2);
            \node at (0,0) {\resizebox{1cm}{!}{\input{document.tex}}};
            \node at (0,0) {\LaTeX};
            \draw[->] (.6,0) -- (1.4,0);

            \node at (2,0) {\resizebox{1cm}{!}{\input{server.tex}}};
            \draw[->] (2.6,0) -- (3.4,0);
            \node at (4,0) {\resizebox{1cm}{!}{\input{computer.tex}}};
            \node at (-2,-1) {Various};
            \node at (-2,-1.4) {PDFs};
            \node at (0,-1) {Single};
            \node at (0,-1.4) {Source};
            \node at (2,-1) {Server};
            \node at (4,-1) {Students}; 
            \node at (4,-1.4) {Engage};

        \end{tikzpicture}
    \end{center}
    The same source code used to produce a PDF can be deployed to a Ximera server, where it can be accessed by students.
    To get started as an author in Ximera, all you need is the XimeraLaTeX
    Package, which is available on CTAN.
    \\[.5cm]
    {\sffamily\bfseries Instructors}\\
    Instructors (who are not authors) can freely use any Ximera materials,
    \textbf{without permission}, simply by using the URL of the
    course. See our website for an incomplete list of Ximera
    courses that have been
    deployed online.
    \\[.5cm]
    {\sffamily\bfseries Students}\\
    Ximera materials are free. Students can use any of the materials they
    find, even if the student is not enrolled in a course.
    \\[.5cm]
    {\sffamily\bfseries Current Development}\\
    Ximera is maintained by a community of people and we are always working. In
    particular: We are working to streamline the deploy process,
    make LTI 1.3 support widely available, and provide and accessible experience.
    \begin{center}
        \begin{tikzpicture}
            \node at (-3,1) [cloud, draw,cloud puffs=10,cloud puff arc=120, aspect=2, inner ysep=1em,scale=.8] {};
            \node at (-3,1) {LMS};
            \node at (-3,-1) {\resizebox{1cm}{!}{\input{computer.tex}}};

            \node at (0,0) {\resizebox{1cm}{!}{\input{modulus.tex}}};

            \node at (3,-1) {\resizebox{1cm}{!}{\input{server.tex}}};
            \node at (3,1) {\resizebox{1cm}{!}{\input{computer.tex}}};


        \end{tikzpicture}
    \end{center}



    If you are interested in using Ximera or contributing to the effort, either as an instrutor, an author, or a developer, consider
    applying for a \textbf{Ximera Flash-Grant Stipend} here:
    \begin{center}
      \tt  https://go.osu.edu/ximera-flash-grant/
    \end{center}
    Thank you for your interest in Ximera. We encouage you
    to contact the team with  any question you may have.
    \\[.5cm]
    {\sffamily\bfseries Talks at \textsl{MathFest} with Ximera}\\
    \begin{itemize}
        \item[{[1]}] \textbf{Lines of Sight: Activities Related to Visual
            Perspective}. A.\ Davis; Thursday, August 8th, 10:20am--10:35am, Room 309/310,
        \textit{New Twists on Your Favorite Math Circle Activity Part A}.
        \item[{[2]}] \textbf{Ximera in the Classroom}. B.\ Snapp and J.\ Fowler;
        Friday, August 9th, 3:40pm--3:55pm, Room 313, \textit{Open-Source Products for
            the Advancement of Math Education Research and Practice}.
    \end{itemize}

\end{multicols}
\end{document}
