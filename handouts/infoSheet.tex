\documentclass{article}
\usepackage{tikz}
\usepackage{standalone}
\usepackage[cm]{fullpage}
\usepackage{courier}
\usepackage{helvet}


\usepackage{multicol}

\usepackage[letterspace=-100]{microtype}
\begin{document}
\pagenumbering{gobble}
\sffamily\flushleft
\hspace{-.3cm}\begin{tikzpicture}%
    \node at (0,0) {\tikzset{draw/.append style={black}}
    \resizebox{!}{.05\textheight}{\input{../wormLogo/tikzLogo.tex}}};
    \node at (6.5,0) {\resizebox{!}{.05\textheight}{\lsstyle InfoSheet}};
\end{tikzpicture}\\
Ximera: Interactive Mathematics Education \\Resources for All\\[.1cm]
\rule{\textwidth}{.1cm}\\[.1cm]\ttfamily
Email: ximera@math.osu.edu \hfill Website:
https://github.com/ximeraProject/\\[1cm]
\begin{multicols}{2}
    Ximera, pronounced "chimera," is an
    open-source platform that provides tools for authoring and publishing
    (PDF and Online), open-source, interactive educational content, such
    as textbooks, assessments, and online courses.
    The Ximera Project is
    funded 2024-2026 with (no other external funding) by a U.S.\ Department of
    Education \$2,125,000
    Open Textbooks Pilot
    Program grant. In the
    past, the Ximera Project has also recieved support from NSF Grant
    DUE-1245433, the Shuttleworth Foundation, the Ohio State University
    Department of Mathematics, and the Affordable Learning Exchange at
    OSU.\\[.5cm]
    {\sffamily\bfseries Authors}\\
    With Ximera, authors use LaTeX to create their content. With this
    single source code, we generate different types of output: A PDF worksheet, an
    online interactive activity, and/or a PDF solution manual.
    \begin{center}
        \includegraphics[width=.4\textwidth]{SimultaneousOutput.jpg}
    \end{center}
    \begin{center}
        \begin{tikzpicture}
            % \node at (-3,1) {\resizebox{.5cm}{!}{\input{document.tex}}};
            % \node at (-3,1) {\tiny PDF};
            % \node at (-3,0) {\resizebox{.5cm}{!}{\input{document.tex}}};
            % \node at (-3,0) {\tiny PDF};
            % \node at (-3,-1) {\resizebox{.5cm}{!}{\input{document.tex}}};
            % \node at (-3,-1) {\tiny PDF};
            \node at (-2.8,.2) {\resizebox{.5cm}{!}{\input{document.tex}}};
            \node at (-2.8,.2) {\tiny PDF};
            \node at (-3,0) {\resizebox{.5cm}{!}{\input{document.tex}}};
            \node at (-3,0) {\tiny PDF};
            \node at (-3.2,-.2) {\resizebox{.5cm}{!}{\input{document.tex}}};
            \node at (-3.2,-.2) {\tiny PDF};

            \node at (0,0) {\resizebox{1cm}{!}{\input{document.tex}}};
            \node at (0,0) {\LaTeX};

            \node at (3,0) {\resizebox{1.3cm}{!}{\input{server.tex}}};


        \end{tikzpicture}
    \end{center}
    To get started as an author in Ximera, all you need is the XimeraLaTeX
    Package, which is available on CTAN.
    \\[.5cm]
    {\sffamily\bfseries Instructors}\\
    Instructors (who are not authors) can freely use any Ximera materials,
    \textbf{without permission}, simply by using the URL of the
    course. See our website for an incomplete list of Ximera
    courses that have been
    deployed online.
    \\[.5cm]
    {\sffamily\bfseries Students}\\
    Ximera materials are free. Students can use any of the materials they
    find, even if the student is not enrolled in a course.
    \\[.5cm]
    {\sffamily\bfseries Current Development}\\
    Ximera is maintained by a community of people and we are always working. In
    particular: We are working to streamline the deploy process,
    make LTI 1.3 support widely available, and provide and accessible experience.
    If you are interested in using Ximera or contributing to the effort, consider
    applying for a \textbf{Ximera Flash-Grant Stipend} here:
    \begin{center}
        https://go.osu.edu/ximera-flash-grant/
    \end{center}
    Thank you for your interest in Ximera. We encouage you
    to contact the team with  any question you may have.
    \\[.5cm]
    {\sffamily\bfseries Talks at MathFest with Ximera}\\
    \begin{itemize}
        \item[{[1]}] \textbf{Lines of Sight: Activities Related to Visual
            Perspective}. A.\ Davis; Thursday, August 8th, 10:20am--10:35am, Room 309/310,
        \textit{New Twists on Your Favorite Math Circle Activity Part A}.
        \item[{[2]}] \textbf{Ximera in the Classroom}. B.\ Snapp and J.\ Fowler;
        Friday, August 9th, 3:40pm--3:55pm, Room 313, \textit{Open-Source Products for
            the Advancement of Math Education Research and Practice}.
    \end{itemize}

\end{multicols}
\end{document}
