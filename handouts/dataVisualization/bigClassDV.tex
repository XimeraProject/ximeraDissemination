\documentclass{techbrief}

\title{Data}
\date{\today}
\event{Special Report}

\newcommand{\modulus}{\textsf{Modulus}}

\usepackage{lipsum}

\begin{document}
\newgeometry{margin=1cm,headheight=3cm,includehead,includefoot}
\pagestyle{main}%\onecolumn
\thispagestyle{title}
\pagenumbering{gobble}
\noindent
\lettrine[lines=2]{A}{} promised feature of \modulus\ is rich data visualization---even when in
aggregate.
In this document we will describe issues and needs of instructors of `Big-Math
5000,' a fictitious course of 5000 students.
Issues that are relevant to \modulus\ include:
\begin{enumerate}
    \item There are many `roles' that instructors have: Course Coordinator,
          Lecturer, Teaching Assistant. How do we push this work back to
          Big-Math 5000?
    \item What sort of meaningful aggregate data can we give?
    \item How can we display this data in a meaningful way?
\end{enumerate}
We believe \modulus\ will be capable of dealing with issues like this

\begin{xframe}
    \textbf{The data} for a given assignment that we seek are 
    \begin{description}
        \item[Time to Completion] This is the time it takes to complete an
            activity.
        \item[Attempts to Completion] This would be the total number of answers
            submitted necessary to complete the activity.
        \item[First Attempt] Percent complete on first
            attempt. All assignments are valued
            0--1.So suppose an assignment has 4 questions. If a student gets
            all 4 correct on the first attempt, then this would be 100\%. If
            the
            student gets
            the first question correct on the first attempt, and every other
            question takes
            more attempts, then this is 25\%.
    \end{description}
    This data may be displayed as a heat-map, and will surely need to be
    scaled/truncated.
\end{xframe}

\begin{xframe}
    \textbf{Big-Math 5000} is a coordinated course. This means there are
    `Course Coordinators' who oversee the entire course of 5000 students,
    there are Lecturers who have classes of 200, and there are Teaching
    Assistants who teach sections of 40. 

    They need to be able to `Set-up' a course in \textsl{Canvas}, and then copy
    this course for their 25 lecturers. There are 102 assignments in this
    course, 100 assignments common to all students, 1 unique to each lecturer, and 1 unique to each recitation
    instructor. No mass-editing can be done. Hence they want a single identifier for
    Big-Math 5000,
    \begin{center}\scriptsize
        \texttt{https://modulus.org/12345/https://assignment001}\\
        \texttt{https://modulus.org/12345/https://assignment002}\\
        \texttt{https://modulus.org/12345/https://assignment003}\\
        \dots
        \texttt{https://modulus.org/12345/https://assignment100}\\
    \end{center}
    This identifier will \textbf{not change} between sections of Big-Math 5000,
    nor will it change between semesters or years. If someone has rights to see
    data in \modulus\ for Big-Math 5000, they will be able to see the data for
    the history of the course.

    Moreover, Course Coordinators need to be able to see data for all students
    grouped by lecture and recitation.

    As a \textbf{solution} to these issues, instructors will have a
    personalized
    \modulus\ assignment that they give all of their courses. If Jim Fowler and
    Bart Snapp are Teaching Assistants in Big-Math 5000, then they each give
    their
    students the respective assignments:
    \begin{center}\scriptsize

        \texttt{https://modulus.org/12345/https://ximera.osu.edu/meetJimFowler}\\

        \texttt{https://modulus.org/12345/https://ximera.osu.edu/meetBartSnapp}\\
        \dots
    \end{center}
    And now the course coordinator knows which student is in which section
    simply by the assignments the student submitted.
\end{xframe}

\begin{xframe}
    \textbf{Visualizing} this data will be achieved via a heat map in an student-assignment matrix:
    \begin{center}
        \begin{tikzpicture}
        \node[anchor=north west] at (-.2,.5) {\scriptsize 100 common assignments};
        \node[anchor=north west] at (-.5,-.4) {\rotatebox{90}{\scriptsize 200 anonomous students along the side}};
        \node[anchor=north west] at (0,0) {\includegraphics[width=3cm]{lecture.png}};
        \node[] at (5,-2) {\scriptsize Unique};
        \node[] at (5,-2.3) {\scriptsize recitation assignments};



        \draw[->] (3.8,-2.3) -- (3.2,-.5);
        \draw[->] (3.8,-2.3) -- (3.2,-1.5);
        \draw[->] (3.8,-2.3) -- (3.2,-2.3);
        \draw[->] (3.8,-2.3) -- (3.2,-3.3);
        \draw[->] (3.8,-2.3) -- (3.2,-4.3);
        

        \draw[->] (4,-5.2) -- (2.5,-4.7);
        \node[] at (4,-5.5) {\scriptsize Unique lecture  assignment};

        \end{tikzpicture}
    \end{center}   
    With such a visualization, we would be able to better understand the difficulty of our assignments, and see if differences in performance occur between recitations/lectures.
\end{xframe}

\restoregeometry
\onecolumn\noindent
With this model of visualization, flexible \textbf{sorting options}, ways to \textbf{discard erroneous assignments}, and \textbf{options for the range of the heat map}, we should be able to effectively visualize the entire course, with a mock-up given below:
\begin{center}
\includegraphics[width=3cm]{course1-5.png}\includegraphics[width=3cm]{course6-10.png}\includegraphics[width=3cm]{course11-15.png}\includegraphics[width=3cm]{course16-20.png}\includegraphics[width=3cm]{course21-25.png}
\end{center}
Here we see the data for all assignments for the entire 5000 student course. With this set-up, we would be able to identify problematic assignments and issues in the course.
% \begin{xframe}
%     \textbf{Funding for the Ximera Project} is provided by
%     a U.S.\ Department of Education Open Textbooks Pilot Program grant in the
%     amount of \$2,125,000, from 2024--2026, with no external funding. In the
%     past, the Ximera Project has
%     also received support from NSF Grant DUE-1245433, the Shuttleworth
%     Foundation, the Ohio State University
%     Department of Mathematics, and the Affordable Learning Exchange at OSU.
% \end{xframe}

\end{document}