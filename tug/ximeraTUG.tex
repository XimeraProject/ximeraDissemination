% $Date: 2024/06/04 17:45:48 $
% This template file is public domain.
%
% TUGboat class documentation is at:
%   texdoc tugboat
% or
%   https://texdoc.org/serve/tugboat/0
% or
%   https://mirrors.ctan.org/macros/\LaTeX\/contrib/tugboat/ltubguid.pdf

\documentclass{ltugboat}
\usepackage[T1]{fontenc}
\usepackage{graphicx}
\usepackage{microtype}
\usepackage[hidelinks]{hyperref}

\title{Ximera: A \\LaTeX\-Based Open-Source Platform for Interactive STEM
    Education}

% repeat info for each author; comment out items that don't apply.
\author{First Last}
\address{Street Address \\ Town, Postal \\ Country}
\netaddress{user (at) example dot org}
\personalURL{https://example.org/~user/}
%\ORCID{0}

% Please state if you'd like to receive a physical copy of the TUGboat
% issue or if electronic access suffices.  If you want a physical issue,
% please include the mailing address we should use, as a comment if
% you prefer it not be printed.

\begin{document}
\maketitle

\begin{abstract}
    TO WRITE
\end{abstract}

\section{Introduction}

Ximera is an open-source platform designed to create interactive online educational
content using \LaTeX\ as the source code. Initially developed at The Ohio State
University, Ximera is now used by over a dozen institutions, including
the University of Florida and KU Leuven. By leveraging the semantic structure
of \LaTeX\ and integrating tools like Desmos, GeoGebra, and YouTube, Ximera allows
educators to generate high-quality print materials \emph{and} interactive online
content from the same source.

To be clear, Ximera is designed to create educational content. Our goals are
distinct from those of TeX4ht and MathJax.

\section{Authoring Content in Ximera}


Ximera documents are written in \LaTeX, but the platform extends the standard
\LaTeX\ commands to support interactive elements, such as answer boxes,
multiple-choice questions, and dynamic graphs. Authors write the content on
their own machines using tools such as Visual Studio Code, Docker, and git for
version control. This local setup allows authors to preview both PDF and HTML
outputs before publishing the content online.

Once the Ximera package is installed from CTAN, authors can immediately
generate PDF versions of their materials. However, to publish an online
version, an extra build environment, known as xake, is required. This build
environment compiles \LaTeX\ into HTML and facilitates deployment to a public
server at ximera.osu.edu or to self-hosted servers. The result is a seamless
transition between static print content and dynamic, interactive online
resources.
Document Class Options

Ximera's flexibility extends to how content is displayed. By default, Ximera
documents show all content, including answers and instructor notes. This is
useful for authors during the development phase, as it ensures they can review
the complete content. However, for classroom use, Ximera provides the handout
option, which suppresses answers and interactive elements, making it suitable
for distribution as student worksheets.

For example, the following document would display answers in development mode
but hide them when compiled with the handout option:
\begin{verbatim}
\documentclass[handout]{ximera}
\begin{document}
\begin{problem}
What is the answer to life, the universe, and everything?
\[
    \answer{42}
\]
\end{problem}
\end{document}
\end{verbatim}
This flexibility ensures that the same source document can be used for both
interactive online learning and traditional printed materials.



\section{Abbreviation macros and typing}

The \texttt{ltugboat} class provides many abbreviation commands; here
are a few of the most common:

% verbatim blocks are often better in \small
\begin{verbatim}[\small]
\AllTeX \AMS \BibTeX \Cplusplus \CTAN \DVI
\HTML \\LaTeX\e \macOS \MathML \MF \PDF \PS
\TUB \TUG \tug \WEB \Xe\LaTeX\ \XeTeX \XML
\end{verbatim}

A few other typing conventions:

\begin{itemize}
    \item For an em-dash with our spacing (preferred to \verb|---|):
          \cs{Dash}, with output\Dash like this.

    \item For initialisms in all caps:
          \verb|\acro{FRED}|,\\ with output: \acro{FRED}.

    \item A literal control sequence:
          \verb|\cs{fred}|,\\ with output: \cs{fred}.

    \item A syntactic metavariable:
          \verb|\meta{fred}|,\\ with output: \meta{fred}.

    \item A title:
          \verb|\titleref{Book of Fred}|,\\ with output: \titleref{Book of
              Fred}.
\end{itemize}

We recommend using \cs{begin}\tubbraced{verbatim}\texttt{[\small] ...}
\cs{end}\tubbraced{verbatim} for code blocks, since we prefer not to
colorize code when printed. But if you want to have some font changes,
our recommended settings for the \texttt{listings} package are in the
\texttt{ltubguid} manual mentioned below.

Please put punctuation outside quotes, ``like this'', unless the
punctuation is actually part of the quoted material.

Also, please put punctuation after footnotes.

\section{Figures}

For \TUB, the standard \texttt{figure} environment produces a
column-width figure; this is desirable when at all possible. The
\texttt{figure*} environment produces a full-width (across both columns)
figure when needed. Analogously for \texttt{table} and \texttt{table*}.

Please put captions below figures, but above tables.

Don't worry overmuch about figure placements, as they will likely change
with editing.

\begin{figure}
    This is a column-width figure, made with \\
    \cs{begin}\tubbraced{figure}. Use \tubbraced{figure*} for a full-width
    (double-column) figure.
    %
    \caption{Caption for column-width figure.}
    \label{fig.example}
\end{figure}

%\begin{figure*}
%A full-width figure, made with \cs{begin}\tubbraced{figure*}.
%\caption{Caption for full-width figure.}
%\label{fig.fullwidth}
%\end{figure*}

\section{References}

For references to other issues of \TUB, please use the format
\textsl{volno}:\textsl{issno}, e.g., ``\TUB\ 32:1''.

The primary \TUB\ style documentation is the \texttt{ltubguid} manual,
available online at \tbsurl{ctan.org/pkg/tugboat} or locally with
\texttt{texdoc tugboat}. For general \CTAN\ package references, we
recommend that form, using \texttt{/pkg/}. If you need to refer to a
specific file on \CTAN, use:\\
\texttt{https://mirror.ctan.org/\textsl{path}}

We recommend using \BibTeX\ (but don't require it), with the
\texttt{tugboat} \BibTeX\ style. The \texttt{biblio} directory on \CTAN,
\tbsurl{ctan.org/pkg/biblio}, provides files \texttt{tugboat.bib} with a
complete bibliography of \TUB, \texttt{texbook3.bib} with many common
\TeX-related books and articles, and plenty more.

Email \verb|tugboat@tug.org| with any questions.

\bibliographystyle{tugboat} % tugboat's bibtex style
\nocite{book-minimal}	    % make the example bibliography non-empty
\bibliography{xampl}	    % xampl.bib comes with bibtex

\makesignature
\end{document}
