% $Date: 2024/06/04 17:45:48 $
% This template file is public domain.
%
% TUGboat class documentation is at:
%   texdoc tugboat
% or
%   https://texdoc.org/serve/tugboat/0
% or
%   https://mirrors.ctan.org/macros/\LaTeX\/contrib/tugboat/ltubguid.pdf

\documentclass{ltugboat}
\usepackage[T1]{fontenc}
\usepackage{graphicx}
\usepackage{microtype}
\usepackage[hidelinks]{hyperref}

\title{Ximera: A \LaTeX{-}Based Open-Source Platform for Interactive
    Education}

% repeat info for each author; comment out items that don't apply.
\author{Fowler Jim}
\address{231 W. 18th Ave \\ Columbus, Ohio 43210 \\ USA}
\netaddress{snapp.291@osu.edu}
\personalURL{https://kisonecat.com/}
\author{Nowell Jason}
\address{1400 Stadium Rd \\ Gainesville, Florida 32611 \\ USA}
\netaddress{JNowell@ufl.edu}
\personalURL{https://www.jasonnowell.com/}
\author{Obbels Wim}
\address{Celestijnenlaan 200B \\ 3001 Leuven, \\ Belgium}
\netaddress{wim.obbels@kuleuven.be}
\personalURL{https://www.kuleuven.be/wieiswie/nl/person/00045050}
\author{Snapp Bart}
\address{231 W. 18th Ave \\ Columbus, Ohio 43210 \\ USA}
\netaddress{snapp.14@osu.edu}
\personalURL{https://people.math.osu.edu/snapp.14/}

%\ORCID{0}

% Please state if you'd like to receive a physical copy of the TUGboat
% issue or if electronic access suffices.  If you want a physical issue,
% please include the mailing address we should use, as a comment if
% you prefer it not be printed.

\begin{document}
\maketitle

\begin{abstract}
    TO WRITE
\end{abstract}

\section{Introduction}

Ximera is an open-source platform designed to create interactive online
educational
content using \LaTeX\ as the source code. Initially developed at the Ohio State
University, Ximera is now used by over a dozen institutions, including
the University of Florida and KU Leuven. By leveraging the semantic structure
of \LaTeX\ and integrating tools like Desmos, GeoGebra, and YouTube, Ximera
allows educators to generate high-quality print materials \emph{and}
interactive online content simultaneously from the a single source.

To be clear, Ximera is designed to create \emph{educational content}. Our goals
are distinct from those of \TeX4ht and MathJax. While \TeX4ht seeks to
transform \LaTeX\ into \HTML, there is no additional interactivity. Moreover,
while \TeX4ht attempts to recreate the look of a \PDF\ online,
Ximera makes no such attempt to replicate a print-book experience. Ximera
instead attempts to do something reasonable
online, with the given semantic markup. To this end,
Ximera \emph{uses} \TeX4ht to convert a majority of the content to \HTML. In
turn,
\TeX4ht \emph{uses} MathJax to render mathematics online.

\section{Authoring Content in Ximera}

Ximera documents are written in \LaTeX, with the document class providing
additional
commands supporting interactive elements, such as answer boxes,
multiple-choice questions, and other virtual manipulates.

Authors write the content on their own machines using tools such as Visual
Studio Code, Docker, and git for
version control. This local setup allows authors to preview both PDF and HTML
outputs before publishing the content online.

\subsection{Document Class Options}

Ximera's flexibility extends to how content is displayed. By default, Ximera
documents show all content, including answers and instructor notes. This is
useful for authors during the development phase, as it ensures they can review
the complete content. However, for classroom use, Ximera provides the handout
option, which suppresses answers and interactive elements, making it suitable
for distribution as student worksheets.

For example, the following document would display answers in development mode
but hide them when compiled with the handout option:
\begin{verbatim}[\small] 
\documentclass[handout]{ximera}
\begin{document}
\begin{problem}
What is the answer to life, the universe, 
and everything?
\begin{solution}
\[
    \answer{42}
\]
\end{solution}
\end{problem}
\end{document}}
\end{verbatim}
This flexibility ensures that the same source document can be used for both
interactive online learning and traditional printed materials.

NOTE THE NESTING OF ENVIRONMENTS!

\subsection{Interactive Elements and Features}

Ximera provides several \LaTeX\ commands to create interactive elements within
documents, enhancing student engagement. These commands include:

\cs{answer}: Allows students to input answers directly in the online platform.
For example:

\begin{verbatim}[\small]
\begin{problem}
Differentiate $f(x) = x^2$.
\[
    \answer{2x}
\]
\end{problem}
\end{verbatim}

\verb|choice| environments: Used to create multiple-choice and “select all that
apply” questions, respectively. These questions can be graded automatically,
providing instant feedback to students.

\subsection{Graphics} NEEDS TO BE WRITTEN: GRAPHICS PATHS!

and Visualization: Ximera integrates with external tools like Desmos
and GeoGebra to create dynamic, interactive graphs that students can manipulate
in real time. Embedding a Desmos graph, for example, is as simple as:

\begin{verbatim}[\small]
\begin{center}
    \desmos{zwywds7med}{800}{600}
\end{center}
\end{verbatim}
Ximera also supports embedding YouTube videos, allowing educators to integrate
multimedia resources into their lessons seamlessly. These interactive features
enrich the learning experience, making complex STEM concepts more accessible
and engaging for students.

\section{Deploying Ximera content online}

Deploying Ximera courses involves setting up a backend infrastructure that
allows seamless transition from development to publication. Ximera relies
heavily on Git for version control and Docker for consistent build
environments. Here's an overview of how deployment works.
The Role of Docker in Deployment

Docker is a crucial part of the Ximera deployment process, enabling users to
manage dependencies and software versions consistently across platforms. Ximera
uses Docker containers to ensure that all required software (e.g., LATEX) is
available for compiling documents and deploying them online.

To get started, users must install Docker on their machines. Docker allows
authors to test and compile their content locally before deploying it to the
public Ximera server at ximera.osu.edu or a self-hosted server. Once Docker is
set up, users can interact with it through Visual Studio Code, using a built-in
terminal to execute commands for compiling and deploying Ximera documents.
Using Xake to ``Bake'' Ximera Documents

The Ximera build system, known as xake, automates the process of compiling
LATEX documents into both PDFs and interactive HTML files. The first time a
document is compiled, Xake downloads the required Docker containers and
processes the content. After the initial run, subsequent builds are much
faster, only compiling updated files.

To start the build process, users simply press the ``Bake'' button in Visual
Studio Code or use the following command in the terminal:

bash

./scripts/xmlatex -i bash

Once compiled, the content is ready for deployment.
Deploying to a Ximera Server

After "baking" the content with Xake, the next step is to deploy the course
online. Ximera allows deployment to either a public server hosted by OSU or an
institution’s self-hosted server. To ensure security, Ximera requires users to
have GPG keys for authentication. These keys ensure that only authorized users
can deploy or update courses.

To deploy a course, users need to edit the .ximeraserve file in the repository.
This file contains important information such as the deployment URL and GPG
key. The following is an example configuration in .ximeraserve:


\section{Ongoing Development}

BETTER DESCRIPTION

Ximera is constantly evolving. New features in development include a Docker
setup for easier deployment and a serverless option that enables document
compilation directly in the browser. The Ximera team is also working on
expanding gradebook functionality and adding more tutorials and examples to
help educators get started with the platform.

\bibliographystyle{tugboat} % tugboat's bibtex style
\nocite{book-minimal}	    % make the example bibliography non-empty
\bibliography{xampl}	    % xampl.bib comes with bibtex

\makesignature
\end{document}
